\chapter*{Samenvatting}

Dodona is een online platform om programmeeroefeningen op te lossen en voorziet oplossingen in real time van feedback, bijvoorbeeld over de correctheid van de ingediende oplossing.
Veel oefeningen zijn niet inherent gebonden aan een programmeertaal en zouden opgelost kunnen worden in meerdere programmeertalen.
Het manueel configureren van oefeningen voor verschillende programmeertalen is echter een tijdrovende bezigheid.
In deze masterproef focussen we daarom op de onderzoeksvraag of het mogelijk is om oefeningen slechts één keer op een programmeertaalonafhankelijke manier te configureren en toch oplossingen in verschillende programmeertalen te beoordelen.
Als antwoord hierop introduceert de masterproef \tested{}, een prototype van een judge voor Dodona, die dezelfde oefening in meerdere programmeertalen kan beoordelen.

Een kernaspect van \tested{} is het testplan: een specificatie van hoe een oplossing voor een oefening moet beoordeeld worden.
Bij bestaande judges in Dodona gebeurt dit op een programmeertaalafhankelijke manier: jUnit in Java, doctests in Python, enz.
Het testplan bij \tested{} is programmeertaalonafhankelijk en wordt opgesteld in \acronym{JSON}.
Op deze manier wordt één specificatie opgesteld, waarna de oefening kan opgelost worden in alle programmeertalen die \tested{} ondersteunt.
Momenteel zijn dat Python, Java, Haskell, C en JavaScript.

Een testplan bestaat uit een reeks testgevallen, waarin telkens een invoer gekoppeld wordt aan een uitvoer.
Als invoer ondersteunt het testplan \texttt{stdin}, functieoproepen, commandoargumenten en bestanden.
De beoordeelde uitvoer bestaat uit \texttt{stdout}, \texttt{stderr}, exceptions, returnwaarden, aangemaakte bestanden en de exitcode.
Een programmeertaalonafhankelijk testplan betekent niet dat we geen specificatie kunnen schrijven die bepaalde functionaliteit gebruikt die niet aanwezig is in elke programmeertaal.
Een voorbeeld is een testplan met klassen, dat enkel opgelost zal kunnen worden in objectgericht programmeertalen.

Voor de vertaling van het testplan naar testcode in de programmeertaal van de oplossing gebruikt \tested{} een sjabloonsysteem genaamd Mako, gelijkaardig aan sjabloonsystemen die gebruikt worden bij webapplicaties (\acronym{ERB} bij Rails, Blade bij Laravel, enz.).
Deze sjablonen worden eenmalig opgesteld bij de configuratie van een programmeertaal in \tested{}.
Ze worden bijvoorbeeld gebruikt voor het vertalen van expressies en statements uit het testplan.

We trekken volgende conclusies uit de ontwikkeling van \tested{}:

\begin{enumerate}
    \item Het opstellen van een programmeertaalonafhankelijke specificatie voor een oefening is doenbaar, zonder verlies aan functionaliteit voor de oefeningen die beoordeeld kunnen worden en zonder significant performantieverlies tijdens het beoordelen.
    \item Nieuwe programmeertalen kunnen snel toegevoegd worden aan \tested{}, zonder impact op bestaande oefeningen.
    \item We zijn van mening dat het de moeite loont om \tested{} uit te bouwen tot een volwaardige judge voor Dodona en te onderzoeken hoe we aan Dodona ondersteuning voor oefeningen in meerdere programmeertalen kunnen toevoegen.
\end{enumerate}
