%! Suppress = EscapeHashOutsideCommand
\chapter{Echo-oefening}\label{ch:echo-oefening}

Deze bijlage bevat de door \tested{} gegenereerde testcode voor het testplan uit de oefening \emph{Echo} uit \cref{ch:nieuwe-oefening} (zie \cref{subsec:oefening-echo-testplan} voor het testplan.)

\section{Python}\label{sec:echo-python}

\subsection{Oplossing}\label{subsec:echo-python-oplossing}

\inputminted{python}{sources/echo/correct.py}

\subsection{Gegenereerde code}\label{subsec:echo-python-gegenereerde-code}

De testcode die gegenereerd wordt bij het uitvoeren van bovenstaande testplan.
Deze code is lichtjes aangepast: overtollige witruimte is verwijderd.
Anders is de code identiek aan de door \tested{} gegenereerde code.

In Python is geen selector nodig, waardoor de gegenereerde code identiek is voor batchcompilatie en voor contextcompilatie.
In beide gevallen worden twee bestanden gegenereerd:

\begin{enumerate}
    \item \texttt{context\_0\_0.py}
    \item \texttt{context\_0\_1.py}
\end{enumerate}

\subsubsection{\texttt{context\_0\_0.py}}

%! Suppress = EscapeUnderscore
\inputminted{python}{sources/echo/context_0_0.py}

\subsubsection{\texttt{context\_0\_1.py}}

%! Suppress = EscapeUnderscore
\inputminted{python}{sources/echo/context_0_1.py}

\subsection{Uitvoeren}\label{subsec:echo-python-uitvoeren}

Bij het uitvoeren worden de gegenereerde bestanden uitgevoerd.
In beide gevallen (batchcompilatie en contextcompilatie) wordt de juiste context uitgevoerd:

%! Suppress = EscapeUnderscore
\begin{minted}{console}
> python context_0_0.py
<uitvoer context_0_0>
> python context_0_1.py
<uitvoer context_0_1>
\end{minted}

\section{Java}\label{sec:echo-java}

\subsection{Oplossing}\label{subsec:echo-java-oplossing}

\inputminted{java}{sources/echo/correct.java}

\subsection{Gegenereerde code}\label{subsec:echo-java-gegenereerde-code}

De testcode die gegenereerd wordt bij het uitvoeren van bovenstaande testplan.
Deze code is lichtjes aangepast: overtollige witruimte is verwijderd.
Anders is de code identiek aan de door \tested{} gegenereerde code.

Bij batchcompilatie worden alle testgevallen in één keer gecompileerd.
Daarvoor wordt het bestand \texttt{Selector.java} gegenereerd.
Bij contextcompilatie wordt geen \texttt{Selector.java} gegenereerd, maar zijn de bestanden anders identiek.
Volgende bestanden werden gegenereerd (met \texttt{Selector.java} dus enkel in batchcompilatie):

\begin{enumerate}
    \item \texttt{Context00.java}
    \item \texttt{Context01.java}
    \item \texttt{Selector.java}
\end{enumerate}

\subsubsection{\texttt{Context00.java}}

\inputminted{java}{sources/echo/Context00.java}

\subsubsection{\texttt{Context01.java}}

\inputminted{java}{sources/echo/Context01.java}

\subsubsection{\texttt{Selector.java}}

\inputminted{java}{sources/echo/Selector.java}

\subsection{Uitvoeren}\label{subsec:echo-java-uitvoeren}

Bij batchcompilatie wordt \texttt{Selector.java} gecompileerd, wat leidt tot een reeks \texttt{.class}-bestanden.
Vervolgens wordt de selector tweemaal uitgevoerd:

%! Suppress = EscapeUnderscore
%! Suppress = LineBreak
\begin{minted}{console}
> ./java -cp . Selector Context00
<uitvoer context_0_0>
> ./java -cp . Selector Context01
<uitvoer context_0_1>
\end{minted}

Een opmerking hierbij teneinde verwarring tegen te gaan: in de commando's hierboven wordt met het argument \texttt{Selector} de klasse meegegeven aan Java die uitgevoerd moet worden.
Het argument \texttt{Context0*} is een commandoargument zoals gebruikelijk.

Bij contextcompilatie wordt elke context afzonderlijk gecompileerd, wat ook leidt tot een reeks \texttt{.class}-bestanden.
Deze keer worden de contexten zelf uitgevoerd:

%! Suppress = EscapeUnderscore
%! Suppress = LineBreak
\begin{minted}{console}
> java -cp . Context00
<uitvoer context_0_0>
> java -cp . Context01
<uitvoer context_0_1>
\end{minted}

\section{C}\label{sec:echo-c}

\subsection{Oplossing}\label{subsec:echo-c-oplossing}

\inputminted{c}{sources/echo/correct.c}

\subsection{Gegenereerde code}\label{subsec:echo-c-gegenereerde-code}

De testcode die gegenereerd wordt bij het uitvoeren van bovenstaande testplan.
Deze code is lichtjes aangepast: overtollige witruimte is verwijderd.
Anders is de code identiek aan de door \tested{} gegenereerde code.

Bij batchcompilatie worden alle testgevallen in één keer gecompileerd.
Daarvoor wordt het bestand \texttt{selector.c} gegenereerd.
Bij contextcompilatie wordt geen \texttt{selector.c} gegenereerd, maar zijn de bestanden anders identiek.
Volgende bestanden werden gegenereerd (met \texttt{selector.c} dus enkel in batchcompilatie):

\begin{enumerate}
    \item \texttt{context\_0\_0.c}
    \item \texttt{context\_0\_1.c}
    \item \texttt{selector.c}
\end{enumerate}

\subsubsection{\texttt{context\_0\_0.c}}

%! Suppress = EscapeUnderscore
\inputminted{c}{sources/echo/context_0_0.c}

\subsubsection{\texttt{context\_0\_1.c}}

%! Suppress = EscapeUnderscore
\inputminted{c}{sources/echo/context_0_1.c}

\subsubsection{\texttt{selector.c}}

\inputminted{c}{sources/echo/selector.c}

\subsection{Uitvoeren}\label{subsec:echo-c-uitvoeren}

Bij batchcompilatie wordt \texttt{selector.c} gecompileerd, wat leidt tot een uitvoerbaar bestand \texttt{selector}.
Dit laatste bestand wordt dan tweemaal uitgevoerd:

%! Suppress = EscapeUnderscore
\begin{minted}{console}
> ./selector context_0_0
<uitvoer context_0_0>
> ./selector context_0_1
<uitvoer context_0_1>    
\end{minted}

Bij contextcompilatie wordt elke context afzonderlijk gecompileerd, wat leidt tot twee uitvoerbare bestanden: \texttt{context\_0\_0} en \texttt{context\_0\_1}.
Deze uitvoerbare bestanden worden vervolgens uitgevoerd:

%! Suppress = EscapeUnderscore
\begin{minted}{console}
> ./context_0_0
<uitvoer context_0_0>
> ./context_0_1
<uitvoer context_0_1>
\end{minted}


\section{Haskell}\label{sec:echo-haskell}

\subsection{Oplossing}\label{subsec:echo-haskell-oplossing}

\inputminted{haskell}{sources/echo/correct.hs}

\subsection{Gegenereerde code}\label{subsec:echo-haskell-gegenereerde-code}

De testcode die gegenereerd wordt bij het uitvoeren van bovenstaande testplan.
Deze code is lichtjes aangepast: overtollige witruimte is verwijderd.
Anders is de code identiek aan de door \tested{} gegenereerde code.

Bij batchcompilatie worden alle testgevallen in één keer gecompileerd.
Daarvoor wordt het bestand \texttt{Selector.hs} gegenereerd.
Bij contextcompilatie wordt geen \texttt{Selector.hs} gegenereerd, maar zijn de bestanden anders identiek.
Volgende bestanden werden gegenereerd (met \texttt{Selector.hs} dus enkel in batchcompilatie):

\begin{enumerate}
    \item \texttt{Context00.hs}
    \item \texttt{Context01.hs}
    \item \texttt{Selector.hs}
\end{enumerate}

Op het einde van alle contexten staat in de Haskell-code \mintinline{haskell}{putStr ""}.
Dit is met opzet en nodig omdat de \texttt{do}-notatie in Haskell moet eindigen met een expressie.

\subsubsection{\texttt{Context00.hs}}

\inputminted{haskell}{sources/echo/Context00.hs}

\subsubsection{\texttt{Context01.hs}}

\inputminted{haskell}{sources/echo/Context01.hs}

\subsubsection{\texttt{Selector.hs}}

\inputminted{haskell}{sources/echo/Selector.hs}

\subsection{Uitvoeren}\label{subsec:echo-haskell-uitvoeren}

Bij batchcompilatie wordt \texttt{Selector.hs} gecompileerd, wat leidt tot een uitvoerbaar bestand \texttt{selector}.
Dit laatste bestand wordt dan tweemaal uitgevoerd:

%! Suppress = EscapeUnderscore
\begin{minted}{console}
> ./selector Context00
<uitvoer context_0_0>
> ./selector Context01
<uitvoer context_0_1>
\end{minted}

Bij contextcompilatie wordt elke context afzonderlijk gecompileerd, wat leidt tot twee uitvoerbare bestanden: \texttt{Context00} en \texttt{Context01}.
Deze uitvoerbare bestanden worden vervolgens uitgevoerd:

%! Suppress = EscapeUnderscore
\begin{minted}{console}
> ./Context00
<uitvoer context_0_0>
> ./Context01
<uitvoer context_0_1>
\end{minted}

\section{JavaScript}\label{sec:echo-javascript}

\subsection{Oplossing}\label{subsec:echo-javascript-oplossing}

\inputminted{python}{sources/echo/correct.js}

\subsection{Gegenereerde code}\label{subsec:echo-javascript-gegenereerde-code}

De testcode die gegenereerd wordt bij het uitvoeren van bovenstaande testplan.
Deze code is lichtjes aangepast: overtollige witruimte is verwijderd.
Anders is de code identiek aan de door \tested{} gegenereerde code.

In JavaScript is geen selector nodig, waardoor de gegenereerde code identiek is voor batchcompilatie en voor contextcompilatie.
In beide gevallen worden twee bestanden gegenereerd:

\begin{enumerate}
    \item \texttt{context00.js}
    \item \texttt{context01.js}
\end{enumerate}

\subsubsection{\texttt{context00.js}}

\inputminted{javascript}{sources/echo/context00.js}

\subsubsection{\texttt{context01.js}}

\inputminted{python}{sources/echo/context01.js}

\subsection{Uitvoeren}\label{subsec:echo-javascript-uitvoeren}

Bij het uitvoeren worden de gegenereerde bestanden uitgevoerd.
In beide gevallen (batchcompilatie en contextcompilatie) wordt de juiste context uitgevoerd:

%! Suppress = EscapeUnderscore
\begin{minted}{console}
> node context00.js
<uitvoer context_0_0>
> node context01.js
<uitvoer context_0_1>
\end{minted}