\chapter{Echo-oefening}\label{ch:echo-oefening}

In deze bijlage is een volledige oefening beschikbaar, met opgave, testplan en de gegenereerde code.
Om de codefragmenten enigszins kort te houden, zijn er maar twee contexten in de oefening.

\section{Opgave}\label{sec:echo-opgave}

De opgave van deze oefening luidt als volgt:

\begin{quote}
    \markdownInput[renderers = {
        headingOne = {\chapter*{#1}},
        headingTwo = {\section*{#1}},
        headingThree = {\subsection*{#1}},
    }]{sources/echo-c/description.md}
\end{quote}

\section{Testplan}\label{sec:echo-testplan}

Een testplan met twee testgevallen, elk in een andere context:

\inputminted{json}{sources/echo-c/two.tson}

\section{Oplossing}\label{sec:echo-oplossing}

We voeren het testplan uit met deze oplossing:

\inputminted{c}{sources/echo-c/correct.c}

\section{Gegenereerde code}\label{sec:echo-gegenereerde-code}

De gegenereerde code die gegenereerd wordt bij het uitvoeren van bovenstaande testplan.
Deze code is lichtjes aangepast: overtollige witruimte is verwijderd.
Anders is de code identiek aan de door TESTed gegenereerde code.

\subsection{Batchcompilatie}\label{subsec:echo-batchcompilatie}

In batchcompilatie worden alle testgevallen in één keer gecompileerd.
Volgende bestanden werden gegenereerd:

\begin{enumerate}
    \item \texttt{context\_0\_0.c}
    \item \texttt{context\_0\_1.c}
    \item \texttt{selector.c}
\end{enumerate}

\subsubsection{\texttt{context\_0\_0.c}}

\inputminted{c}{sources/echo-c/context_0_0.c}

\subsubsection{\texttt{context\_0\_1.c}}

\inputminted{c}{sources/echo-c/context_0_1.c}

\subsubsection{\texttt{selector.c}}

\inputminted{c}{sources/echo-c/selector.c}

\subsubsection{Uitvoeren}

Bij het compileren wordt \texttt{selector.c} gecompileerd, wat leidt tot een uitvoerbaar bestand \texttt{selector}.
Dit laatste bestand wordt dan tweemaal uitgevoerd:

\begin{minted}{console}
> ./selector context_0_0
<uitvoer context_0_0>
> ./selector context_0_1
<uitvoer context_0_1>    
\end{minted}

\subsection{Contextcompilatie}\label{subsec:echo-contextcompilatie}

Bij contextcompilatie worden slechts twee bestanden gegenereerd:

\begin{enumerate}
    \item \texttt{context\_0\_0.c}
    \item \texttt{context\_0\_1.c}
\end{enumerate}

De inhoud van deze bestanden is identiek aan de inhoud van dezelfde bestanden bij batchcompilatie.

Elk van deze bestanden wordt afzonderlijk gecompileerd, wat leidt tot twee uitvoerbare bestanden: \texttt{context\_0\_0} en \texttt{context\_0\_1}.
Deze uitvoerbare bestanden worden vervolgens uitgevoerd:

\begin{minted}{console}
> ./context_0_0
<uitvoer context_0_0>
> ./context_0_1
<uitvoer context_0_1>
\end{minted}
