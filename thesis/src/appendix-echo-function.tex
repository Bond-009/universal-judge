%! Suppress = EscapeHashOutsideCommand
\chapter{Echofunctie-oefening}\label{ch:echo-function-oefening}

Deze bijlage bevat de door \tested{} gegenereerde testcode voor het testplan uit de oefening \emph{Echofunctie} uit \cref{ch:nieuwe-oefening} (zie \cref{subsec:oefening-echofunctie-testplan} voor het testplan.)

\section{Python}\label{sec:echo-function-python}

\subsection{Oplossing}\label{subsec:echo-function-python-oplossing}

\inputminted{python}{sources/echo-function/correct.py}

\subsection{Gegenereerde code}\label{subsec:echo-function-python-gegenereerde-code}

De testcode die gegenereerd wordt bij het uitvoeren van bovenstaande testplan.
Deze code is lichtjes aangepast: overtollige witruimte is verwijderd.
Anders is de code identiek aan de door \tested{} gegenereerde code.

In Python is geen selector nodig, waardoor de gegenereerde code identiek is voor batchcompilatie en voor contextcompilatie.
In beide gevallen wordt een bestanden gegenereerd:

\begin{enumerate}
    \item \texttt{context\_0\_0.py}
\end{enumerate}

\subsubsection{\texttt{context\_0\_0.py}}

%! Suppress = EscapeUnderscore
\inputminted{python}{sources/echo-function/context_0_0.py}

\subsection{Uitvoeren}\label{subsec:echo-function-python-uitvoeren}

Bij het uitvoeren worden de gegenereerde bestanden uitgevoerd.
In beide gevallen (batchcompilatie en contextcompilatie) wordt de juiste context uitgevoerd:

%! Suppress = EscapeUnderscore
\begin{minted}{console}
> python context_0_0.py
<uitvoer context_0_0>
\end{minted}

\section{Java}\label{sec:echo-function-java}

\subsection{Oplossing}\label{subsec:echo-function-java-oplossing}

\inputminted{java}{sources/echo-function/correct.java}

\subsection{Gegenereerde code}\label{subsec:echo-function-java-gegenereerde-code}

De testcode die gegenereerd wordt bij het uitvoeren van bovenstaande testplan.
Deze code is lichtjes aangepast: overtollige witruimte is verwijderd.
Anders is de code identiek aan de door \tested{} gegenereerde code.

Bij batchcompilatie worden alle testgevallen in één keer gecompileerd.
Daarvoor wordt het bestand \texttt{Selector.java} gegenereerd.
Bij contextcompilatie wordt geen \texttt{Selector.java} gegenereerd, maar zijn de bestanden anders identiek.
Volgende bestanden werden gegenereerd (met \texttt{Selector.java} dus enkel in batchcompilatie):

\begin{enumerate}
    \item \texttt{Context00.java}
    \item \texttt{Selector.java}
\end{enumerate}

\subsubsection{\texttt{Context00.java}}

\inputminted{java}{sources/echo-function/Context00.java}

\subsubsection{\texttt{Selector.java}}

\inputminted{java}{sources/echo-function/Selector.java}

\subsection{Uitvoeren}\label{subsec:echo-function-java-uitvoeren}

Bij batchcompilatie wordt \texttt{Selector.java} gecompileerd, wat leidt tot een reeks \texttt{.class}-bestanden.
Vervolgens wordt de selector uitgevoerd:

%! Suppress = EscapeUnderscore
%! Suppress = LineBreak
\begin{minted}{console}
> ./java -cp . Selector Context00
<uitvoer context_0_0>
\end{minted}

Bij contextcompilatie wordt elke context afzonderlijk gecompileerd, wat ook leidt tot een reeks \texttt{.class}-bestanden.
Deze keer wordt de context zelf uitgevoerd:

%! Suppress = EscapeUnderscore
%! Suppress = LineBreak
\begin{minted}{console}
> java -cp . Context00
<uitvoer context_0_0>
\end{minted}

\section{C}\label{sec:echo-function-c}

\subsection{Oplossing}\label{subsec:echo-function-oplossing}

We voeren het testplan uit met deze oplossing:

\inputminted{c}{sources/echo-function/correct.c}

\subsection{Gegenereerde code}\label{subsec:echo-function-c-gegenereerde-code}

De gegenereerde code die gegenereerd wordt bij het uitvoeren van bovenstaande testplan.
Deze code is lichtjes aangepast: overtollige witruimte is verwijderd.
Anders is de code identiek aan de door \tested{} gegenereerde code.

Bij batchcompilatie worden alle testgevallen in één keer gecompileerd.
Volgende bestanden werden gegenereerd:

\begin{enumerate}
    \item \texttt{context\_0\_0.c}
    \item \texttt{selector.c}
\end{enumerate}

\subsubsection{\texttt{context\_0\_0.c}}

%! Suppress = EscapeUnderscore
\inputminted{c}{sources/echo-function/context_0_0.c}

\subsubsection{\texttt{selector.c}}

\inputminted{c}{sources/echo-function/selector.c}

\subsection{Uitvoeren}\label{subsec:echo-function-c-uitvoeren}

Bij het compileren wordt \texttt{selector.c} gecompileerd, wat leidt tot een uitvoerbaar bestand \texttt{selector}.
Dit laatste bestand wordt dan uitgevoerd:

%! Suppress = EscapeUnderscore
\begin{minted}{console}
> ./selector context_0_0
<uitvoer context_0_0>
\end{minted}


Bij contextcompilatie wordt elke context afzonderlijk gecompileerd, wat leidt tot een uitvoerbaar bestand: \texttt{context\_0\_0}.
Dit bestand wordt vervolgens uitgevoerd:

%! Suppress = EscapeUnderscore
\begin{minted}{console}
> ./context_0_0
<uitvoer context_0_0>
\end{minted}
