\chapter{Echofunctie-oefening}\label{ch:echo-function-oefening}

In deze bijlage is een volledige oefening beschikbaar, met opgave, testplan en de gegenereerde code.
Om de codefragmenten enigszins kort te houden, is er maar een context, maar wel met twee testgevallen in dezelfde context.

\section{Opgave}\label{sec:echo-function-opgave}

De opgave van deze oefening luidt als volgt:

\begin{quote}
    \markdownInput[renderers = {
        headingOne = {\chapter*{#1}},
        headingTwo = {\section*{#1}},
        headingThree = {\subsection*{#1}},
    }]{sources/echo-function-c/description.md}
\end{quote}

\section{Testplan}\label{sec:echo-function-testplan}

Een testplan met een context met twee testgevallen:

\inputminted{json}{sources/echo-function-c/one-testcase.tson}

\section{Oplossing}\label{sec:echo-function-oplossing}

We voeren het testplan uit met deze oplossing:

\inputminted{c}{sources/echo-function-c/correct.c}

\section{Gegenereerde code}\label{sec:echo-function-gegenereerde-code}

De gegenereerde code die gegenereerd wordt bij het uitvoeren van bovenstaande testplan.
Deze code is lichtjes aangepast: overtollige witruimte is verwijderd.
Anders is de code identiek aan de door TESTed gegenereerde code.

\subsection{Batchcompilatie}\label{subsec:echo-function-batchcompilatie}

In batchcompilatie worden alle testgevallen in één keer gecompileerd.
Volgende bestanden werden gegenereerd:

\begin{enumerate}
    \item \texttt{context\_0\_0.c}
    \item \texttt{selector.c}
\end{enumerate}

\subsubsection{\texttt{context\_0\_0.c}}

\inputminted{c}{sources/echo-function-c/context_0_0.c}

\subsubsection{\texttt{selector.c}}

\inputminted{c}{sources/echo-function-c/selector.c}

\subsubsection{Uitvoeren}

Bij het compileren wordt \texttt{selector.c} gecompileerd, wat leidt tot een uitvoerbaar bestand \texttt{selector}.
Dit laatste bestand wordt dan uitgevoerd:

\begin{minted}{console}
> ./selector context_0_0
<uitvoer context_0_0>
\end{minted}

\subsection{Contextcompilatie}\label{subsec:echo-function-contextcompilatie}

Bij contextcompilatie wordt slechts een bestand gegenereerd:

\begin{enumerate}
    \item \texttt{context\_0\_0.c}
\end{enumerate}

De inhoud van dit bestanden is identiek aan de inhoud van hetzelfde bestand bij batchcompilatie.

Dit bestand wordt gecompileerd, wat leidt tot een uitvoerbaar bestand: \texttt{context\_0\_0}.
Dit bestand wordt vervolgens uitgevoerd:

\begin{minted}{console}
> ./context_0_0
<uitvoer context_0_0>
\end{minted}
