\chapter{Dodona}\label{ch:dodona}

\section{Inleiding}\label{sec:inleiding}

TODO
Programmeren -> steeds belangrijker en nuttigere kennis om te hebben
Goed programmeren -> vereist veel oefening
Zeker in cursussen met meer mensen -> goede ondersteuning lesgever -> veel tijd
Automatiseren van beoordeling programmeeroefeningen -> Dodona

\section{Wat is Dodona?}\label{sec:wat-is-dodona}

TODO
Intro over Dodona: korte geschiedenis, terminologie, hoe Dodona werkt (oefeningen, judges, enz.)
-> Over de judge wordt in het deel hierna meer verteld.

Dodona:

- Opgaves (of oefeningen), opgesteld door lesgevers
- Oplossingen, opgesteld door studenten
- Judge beoordeelt oplossing van een opgave, door Dodona-team

\section{Evalueren van een oplossing}\label{sec:evalueren-van-een-oplossing}

Zoals vermeld worden de oplossingen van studenten geëvalueerd door een evaluatieprogramma, de \termen{judge}.
In wezen is dit een eenvoudig programma: het krijgt de configuratie via de standaardinvoerstroom (stdin) en schrijft de resultaten van de evaluatie naar de standaarduitvoerstroom (stdout).
Zowel de invoer als de uitvoer van de judge zijn json, waarvan de betekenis vastligt in een json-schema.\footnote{Dit schema en een tekstuele beschrijving is te vinden in de handleiding in~\autocite{dodona2020}.}

De interface opgelegd vanuit Dodona waaraan een judge moet voldoen legt geen beperkingen of vereisten op die verband houden met de programmeertaal.
Via de configuratie krijgt de judge van Dodona enkel in welke programmeertaal de opgave en oplossing zijn.
In de praktijk ondersteunen alle bestaande judges ondersteunen slechts één programmeertaal.
Elke judge ondersteunt oplossingen in de programmeertaal waarin hij geschreven is, m.a.w.\ de Java-judge ondersteunt Java, de Python-judge Python, enz.
Ook heeft elke judge een eigen manier waarop de testen voor een oplossing opgesteld moeten worden.
Zo worden in de Java-judge jUnit-testen gebruikt, terwijl de Python-judge doctests en een eigen formaat ondersteunt.

In grote lijnen verloopt het evalueren van een oplossing van een student als volgt:

\begin{enumerate}
    \item De student dient de oplossing in via de webinterface van Dodona.
    \item Dodona start een Docker-image met de judge.
    \item De judge wordt uitgevoerd, met als invoer de configuratie.
    \item De judge evalueert de oplossing aan de evaluatiecode opgesteld door de lesgever (d.w.z.\ de jUnit-test, de doctests, \ldots).
    \item De judge vertaalt het resultaat van deze evaluatie naar het Dodona-formaat en schrijft dat naar het standaarduitvoerkanaal.
    \item Dodona vangt die uitvoer op, en toont het resultaat aan de student.
\end{enumerate}

\section{Probleemstelling}\label{sec:probleemstelling}

De huidige manier waarop de judges werken resulteert in een belangrijk nadeel.
Bij het bespreken hiervan is het nuttig een voorbeeld in het achterhoofd te houden, teneinde de nadelen te kunnen concretiseren.
Als voorbeeld gebruiken we de "Lotto"-oefening\footnote{Vrij naar een oefening van prof.\ Dawyndt.}, met volgende opgave:

\begin{quotation}
    \markdownInput[slice=^ ^voorbeeld]{\MJmarkdown{code/description.md}}
\end{quotation}

Oplossingen voor deze opgave staan in \cref{lst:java-solution,lst:python-solution}, voor respectievelijk Python en Java.

\begin{listing}
    \inputminted{java}{code/correct-solution.java}
    \caption{Voorbeeldoplossing in Java.}
    \label{lst:java-solution}
\end{listing}

\begin{listing}
    \inputminted{python3}{code/correct-solution.py}
    \caption{Voorbeeldoplossing in Python.}
    \label{lst:python-solution}
\end{listing}

Het belangrijkste nadeel aan de huidige werking is het bijkomende werk voor lesgevers indien zij hun oefeningen in meerdere programmeertalen willen aanbieden.
De Lotto-oefening heeft een eenvoudige opgave en oplossing.
Bovendien zijn de verschillen tussen de versie in Python en Java minimaal, zij het dat de Java-versie wat langer is.
Deze opgave zou zonder problemen in nog vele andere programmeertalen geïmplementeerd kunnen worden.
Deze eenvoudige programmeeroefeningen zijn voornamelijk nuttig in twee gevallen: studenten die voor het eerst leren programmeren en studenten die een nieuwe programmeertaal leren.
In het eerste geval is de eigenlijke programmeertaal minder relevant: het zijn vooral de concepten die belangrijk zijn.
In het tweede geval is de programmeertaal wel van belang, maar moeten gelijkaardige oefeningen gemaakt worden voor elke programmeertaal die aangeleerd moet worden.

We kunnen tot eenzelfde constatatie komen bij ingewikkeldere opgaves die zich concentreren op algoritmen: ook daar zijn de concepten belangrijker dan in welke programmeertaal een algoritme uiteindelijk geïmplementeerd wordt.
Een voorbeeld hiervan is het vak "Algoritmen en Datastructuren" dat gegeven wordt door prof.\ Fack aan de wiskunde\footnote{De studiefiche is voor de geïnteresseerden beschikbaar op \url{https://studiegids.ugent.be/2019/NL/studiefiches/C002794.pdf}}.
Daar zijn de meeste opgaven vandaag al beschikbaar in Java en Python op Dodona, maar dan als afzonderlijke oefeningen.

Het evalueren van een oplossing voor de Lotto-oefening is minder eenvoudig, daar er met willekeurige getallen gewerkt wordt: het volstaat niet om de uitvoer gegenereerd door de oplossing te vergelijken met een op voorhand vastgelegde verwachte uitvoer.
De geproduceerde uitvoer zal moeten gecontroleerd worden met code, specifiek gericht op deze oefening, die de verwachte vereisten van de oplossing controleert.
Deze evaluatiecode moet momenteel voor elke programmeertaal en dus elke judge opnieuw geschreven worden.
In de context van ons voorbeeld controleert deze code bijvoorbeeld of de gegeven getallen binnen het bereik liggen en of ze gesorteerd zijn.

Voor de lesgevers is het opnieuw opstellen van deze evaluatiecode veel en repetitief werk.
Duurt het twee minuten om deze code te schrijven en heeft een vak 30 oefeningen, dan duurt het een uur om alle opgaven op te stellen.
In twee talen duurt dit al twee uur, tien talen vraagt tien uur, enz.

Het probleem hierboven beschreven laat zich samenvatten als volgende onderzoeksvraag, waarop deze thesis een antwoord wil bieden:

\begin{quote}
    Is het mogelijk om een judge zo te implementeren, dat de opgave en evaluatiecode van een oefening slechts eenmaal opgesteld dienen te worden, waarna de oefening beschikbaar is in alle talen die de judge ondersteunt?
    Hierbij willen we dat eens een oefening opgesteld is, deze niet meer gewijzigd moet worden wanneer talen toegevoegd worden aan de judge.
\end{quote}

\section{Opbouw}\label{sec:opbouw}

Het volgende hoofdstuk van deze thesis handelt over het antwoord op bovenstaande vraag.
Daarna volgt ter illustratie een gedetailleerde beschrijving van hoe een nieuwe taal moet toegevoegd worden aan de judge en hoe een opgave kan opgesteld worden voor de judge.
Daar deze twee hoofdstukken voornamelijk ten doel hebben zij die met de judge moeten werken te informeren, nemen deze hoofdstukken de vorm aan van meer traditionele softwarehandleidingen.
Tot slot wordt afgesloten met een hoofdstuk over beperkingen van de huidige implementaties, en waar er verbeteringen mogelijk zijn (het "toekomstige werk").
