%! Suppress = EscapeHashOutsideCommand
%! Suppress = TooLargeSection
\chapter{Educational software testing}\label{ch:dodona}

\lettrine{D}{e evoluties} op technologisch vlak hebben ervoor gezorgd dat onze maatschappij de laatste decennia in hoge mate gedigitaliseerd is, een proces dat nog steeds aan de gang is.
Bovendien kan, door de snelheid waarmee deze veranderingen vaak optreden, eerder gesproken worden van een revolutie dan een evolutie: de veranderingen zijn vaak ingrijpend en veranderen fundamentele aspecten van de sectoren waarin de digitalisering plaatsvindt.
Dit gaat over het ontstaan van nieuwe sectoren, zoals de deeleconomie, maar ook over ingrijpende veranderingen bij bestaande sectoren, zoals de opkomst van \english{ride sharing} in de taxisector.
Ook de impact op maatschappelijk vlak, zoals de sociale media in de politiek, mag niet vergeten worden \autocite{hipeac2019}.

Ook op educatief vlak heeft de digitalisering een grote impact.
Enerzijds biedt digitalisering nieuwe mogelijkheden aan voor onderwijsdoeleinden, zoals het lesgeven op afstand, het online aanbieden van leermateriaal en het online indienen en verbeteren van opdrachten.

Anderzijds biedt het ook uitdagingen: om studenten voor te bereiden op de steeds digitalere maatschappij is een basis van digitale geletterdheid nodig.
Net door de snelle evolutie op technologisch vlak volstaat het niet om studenten te leren werken met de technologie van vandaag;
een grondige kennis van de onderliggende werking van de technologie is onontbeerlijk.

Een belangrijk aspect hierin is het concept van \term{computationeel denken}.
Dat het aanleren van digitale vaardigheden nodig is, bewijst ook de opname van dat computationeel denken in de eindtermen, bijvoorbeeld in het katholieke basisonderwijs \autocite{zinin2017} of in het secundair onderwijs \autocite{2019040867}.

\section{Computationeel denken}\label{sec:computationeel-denken}

Op de vraag wat computationeel denken nu precies betekent lopen de antwoorden uiteen.
Het Departement Onderwijs en Vorming van de Vlaams Overheid \autocite{bastiaensen2017} definieert de term als volgt:

\begin{displayquote}
    Computationeel denken verwijst dus naar het menselijke vermogen om complexe problemen op te lossen en daarbij computers als hulpmiddel te zien.
    Met andere woorden, computationeel denken is het proces waarbij aspecten van informaticawetenschappen herkend worden in de ons omringende wereld, en waarbij de methodes en technieken uit de informaticawetenschappen toegepast worden om problemen uit de fysische en virtuele wereld te begrijpen en op te lossen.
\end{displayquote}

Computationeel denken is dus ruimer dan programmeren, maar programmeren vormt wel een uitstekende manier om het computationeel denken aan te leren en te oefenen.
Bovendien is programmeren op zich ook een nuttige vaardigheid om studenten aan te leren.

\section{Programmeeroefeningen}\label{sec:programmeeroefeningen}

Het aanleren van programmeren is niet eenvoudig en wordt door veel studenten als moeilijk ervaren \autocite{10.1145/3293881.3295779}.
Het maken van oefeningen kan daarbij helpen, indachtig het spreekwoord "oefening baart kunst".
Studenten veel oefeningen laten maken resulteert wel in twee uitdagingen voor de lesgevers:
\begin{enumerate}
    \item Lesgevers moeten geschikte oefeningen opstellen, die rekening houden met welke programmeerconcepten studenten al kennen, tijdslimieten, moeilijkheidsgraden, enzovoort.
    Het opstellen van deze oefeningen vraagt veel tijd.
    \item De oplossingen voor deze oefeningen moeten voorzien worden van kwalitatieve feedback.
    Bij het aanleren van programmeren is feedback een belangrijk aspect om de programmeervaardigheden van de studenten te verbeteren \autocite{10.1145/2899415.2899422}.
\end{enumerate}

Deze masterproef focust op de eerste uitdaging, al wordt ook ingespeeld op de tweede uitdaging, onder andere in \cref{sec:robuustheid}.
De uitdagingen worden beschouwd binnen de context van Dodona, een online leerplatform voor automatische feedback op ingediende oplossingen voor programmeeroefeningen.

\section{Het leerplatform Dodona}\label{sec:wat-is-dodona}

Sinds 2011 wordt aan de onderzoeksgroep Computationele Biologie van de Universiteit Gent gewerkt met programmeeroefeningen waarbij oplossingen in een online systeem ingediend en beoordeeld worden.
Oorspronkelijk werd hiervoor gebruik gemaakt van de \english{Sphere Online Judge} (\acronym{SPOJ}) \autocite{10.1007/978-3-540-78139-4_31}.
Op basis van ervaringen met \acronym{SPOJ} ontwikkelde de onderzoeksgroep een eigen leerplatform, Dodona, dat in september 2016 beschikbaar werd.
Het doel van Dodona is eenvoudig: lesgevers bijstaan om hun studenten niet alleen zo goed mogelijk te leren programmeren, maar dit ook op een zo efficiënt mogelijke manier te doen.

Het online leerplatform Dodona kan opgedeeld worden in verschillende onderdelen:
\begin{enumerate}
    \item Het \textbf{leerplatform} zelf is een webapplicatie, die verantwoordelijk is om de verschillende onderdelen samen te laten werken en die ook de webinterface aanbiedt die de studenten en lesgevers gebruiken.
    Het is via deze interface dat lesgevers oefeningen beschikbaar maken en dat studenten hun oplossingen indienen.
    Het platform zelf is zo opgebouwd dat het onafhankelijk is van de programmeertaal van de oefeningen.
    \item De \textbf{judges} binnen Dodona zijn verantwoordelijk voor het beoordelen van ingediende oplossingen.
    Dit onderdeel wordt uitgebreid besproken in \cref{sec:evalueren-van-een-oplossing}.
    \item De \textbf{oefeningen} worden niet in Dodona zelf aangemaakt, maar worden door de lesgever aangeleverd via een git-repository.
    De oefeningen bevatten de beschrijving van de opgave en de testen die uitgevoerd worden tijdens het automatisch beoordelen van een oplossing.
\end{enumerate}

Met Dodona kunnen lesgevers een leertraject opstellen door een reeks oefeningen te selecteren.
Studenten die dit leertraject volgen, zien onmiddellijk hun voortgang binnen het traject.
Bij het indienen van hun oplossingen ontvangen de studenten ook onmiddellijk feedback over hun oplossing: deze feedback bevat niet alleen de correctheid van de oplossing, maar kan ook andere aspecten belichten, zoals de kwaliteit van de oplossing (onder andere de programmeerstijl), het resultaat van het uitvoeren van linters, en de performantie van de oplossing.

\section{Beoordelen van oplossingen}\label{sec:evalueren-van-een-oplossing}

In Dodona wordt elke ingediende oplossing beoordeeld door een evaluatieprogramma, de \termen{judge}.
In wezen is dit een eenvoudig programma: via de standaardinvoerstroom (\texttt{stdin}) krijgt het programma een configuratie van Dodona.
Deze configuratie bevat onder andere de programmeertaal van de ingediende oplossing, de locatie van de oefeningenbestanden (via de git-repository van de oefening), de locatie van de ingediende oplossing zelf en configuratieopties, zoals geheugen- en tijdslimieten.
Het resultaat van de beoordeling wordt uitgeschreven naar de standaarduitvoerstroom (\texttt{stdout}).
Zowel de invoer als de uitvoer van de judge zijn \acronym{JSON}, waarvan de structuur vastgelegd is in \acronym{JSON} Schema.\footnote{Dit schema en een tekstuele beschrijving ervan is te vinden in de handleiding op \url{https://dodona-edu.github.io/en/guides/creating-a-judge/}.}

Concreet wordt elke beoordeling uitgevoerd in een Docker-container.
Deze Docker-container wordt gemaakt op basis van een Docker-image die bij de judge hoort, en alle dependencies bevat die de judge in kwestie nodig heeft.
Bij het uitvoeren van de beoordeling zal Dodona een \english{bind mount}\footnote{Informatie over deze term is te vinden op \url{https://docs.docker.com/storage/bind-mounts/}} voorzien, zodat de code van de judge zelf, de code van de oefening en de code van de ingediende oplossing beschikbaar zijn in de container.
Via de configuratie geeft Dodona aan de judge mee waar deze bestanden zich bevinden.

Samenvattend bestaat de interface tussen de judge en Dodona uit drie onderdelen:

\begin{enumerate}
    \item De judge zal uitgevoerd worden in een Docker-container, dus een Docker-image met alle dependencies moet voorzien worden.
    Deze Docker-image moet ook de judge opstarten.
    \item Dodona stelt de invoer van een beoordeling ter beschikking aan de judge.
    Bestanden worden via een bind mount aan de Docker-container gekoppeld.
    De paden naar deze bestanden binnen de container en andere informatie (zoals programmeertaal van de oplossing of natuurlijke taal van de gebruiker) worden via de configuratie aan de judge gegeven (via de standaardinvoerstroom).
    \item De judge moet het resultaat van zijn beoordeling uitschrijven naar de standaarduitvoerstroom, in de vastgelegde structuur.
\end{enumerate}

Buiten deze interface legt Dodona geen vereisten op aan de werking van judge.
Door deze vrijheid lopen de manieren waarop de bestaande judges geïmplementeerd zijn uiteen.
Sommige judges beoordelen oplossingen in dezelfde programmeertaal als de taal waarin ze geschreven zijn.
Zo is de judge voor Python-oplossingen geschreven in Python en de judge voor Java-oplossingen in Java.
Bij andere judges is dat niet het geval: de judges voor Bash en Prolog zijn bijvoorbeeld ook in Python geschreven.
Daarnaast heeft elke judge een eigen manier waarop de testen voor de beoordeling van een oplossing opgesteld moeten worden.
Zo worden in de Java-judge jUnit-testen gebruikt, terwijl de Python-judge doctests (en een eigen formaat) ondersteunt.

De beoordeling van een oplossing van een student verloopt als volgt:

\begin{enumerate}
    \item De student dient de oplossing in via de webinterface van Dodona.
    \item Dodona start een Docker-container voor de judge.
    \item Dodona voorziet de container van de bestanden van de judge, de oefening en de ingediende oplossing via de bind mount.
    \item De judge wordt uitgevoerd, waarbij de configuratie via de standaardinvoerstroom beschikbaar is.
    \item De judge beoordeelt de oplossing zoals aangegeven door de lesgever (via jUnit-tests, doctests, \ldots).
    Judges kunnen ook bijkomende taken uitvoeren, zoals linting, beoordeling van de performantie of \english{grading} van de code van de oplossing.
    \item De judge vertaalt zijn beoordeling naar het Dodona-formaat en schrijft het resultaat naar de standaarduitvoerstroom.
    \item Dodona slaat dat resultaat op in de databank.
    \item Op de webinterface krijgt de student het resultaat te zien als feedback op de ingediende oplossing.
\end{enumerate}

\section{Probleemstelling}\label{sec:probleemstelling}

De manier waarop de huidige judges werken resulteert in twee belangrijke nadelen.
Bij het bespreken hiervan is het nuttig een voorbeeld in het achterhoofd te houden, teneinde de nadelen te kunnen concretiseren.
Als voorbeeld gebruiken we de "Lotto"-oefening\footnote{Vrij naar een oefening van prof.\ Dawyndt.
De originele oefening is beschikbaar op \url{https://dodona.ugent.be/nl/exercises/2025591548/}.}, waarvan de opgave hieronder gegeven wordt.
Oplossingen voor deze oefening in twee programmeertalen, Python en Java, staan in \cref{lst:python-solution}.

\begin{quote}
    \markdownInput[renderers = {
        headingOne = {\section*{#1}},
        headingTwo = {\subsection*{#1}},
        headingThree = {\subsubsection*{#1}}
    }, slice=opgave voorbeeld]{sources/lotto-description.md}
\end{quote}

\begin{listing}
    \caption{Oplossing in Python en Java voor de voorbeeldoefening Lotto.}
    \label{lst:python-solution}
    \inputminted{python3}{sources/lotto-correct.py}
    \inputminted{java}{sources/lotto-correct.java}
\end{listing}

\subsection{Opstellen van oefeningen}\label{subsec:opstellen-van-oefeningen}

Het eerste en belangrijkste nadeel aan de werking van de huidige judges heeft betrekking op de lesgevers en komt voor als zij een oefening willen aanbieden in meerdere programmeertalen.
Enerzijds is dit een zware werklast: de oefening, en vooral de code voor de beoordeling, moet voor elke judge opnieuw geschreven worden.
Voor de Python-judge zullen doctests nodig zijn, terwijl de Java-judge jUnit-testen vereist.
Anderzijds lijdt dit ook tot verschillende versies van dezelfde oefening, wat het onderhouden van de oefeningen moelijker maakt.
Als er bijvoorbeeld een fout sluipt in de beoordelingscode, zal de lesgever er aan moeten denken om de fout te verhelpen in alle varianten van de oefening.
Bovendien geeft elke nieuwe versie van de oefening een nieuwe mogelijkheid voor het introduceren van fouten.

Kijken we naar de Lotto-oefening, dan valt op dat het gaat om een eenvoudige opgave en een eenvoudige oplossing.
Bovendien zijn de verschillen tussen oplossingen in verschillende programmeertalen niet zo groot.
In de voorbeeldoplossingen in Python en Java zijn de verschillen minimaal, zij het dat de Java-oplossing wat langer is.
De Lotto-oefening zou zonder problemen in nog vele andere programmeertalen opgelost kunnen worden.
Eenvoudige programmeeroefeningen, zoals de Lotto-oefening, zijn voornamelijk nuttig in twee gevallen: studenten die voor het eerst leren programmeren en studenten die een nieuwe programmeertaal leren.
In het eerste geval is de eigenlijke programmeertaal minder relevant: het zijn vooral de concepten die belangrijk zijn.
In het tweede geval is de programmeertaal wel van belang, maar moeten soortgelijke oefeningen gemaakt worden voor elke programmeertaal die aangeleerd moet worden.
In beide gevallen is het dus een meerwaarde om de oefening in meerdere programmeertalen aan te bieden.

Eenzelfde constatatie kan gemaakt worden bij meer complexe oefeningen die zich concentreren op algoritmen: ook daar zijn de concepten belangrijker dan in welke programmeertaal een algoritme uiteindelijk geïmplementeerd wordt.
Een voorbeeld hiervan is het vak "Algoritmen en Datastructuren", dat gegeven wordt door prof.\ Fack binnen de opleiding wiskunde\footnote{De studiefiche is beschikbaar op \url{https://studiegids.ugent.be/2019/NL/studiefiches/C002794.pdf}}.
Daar zijn de meeste opgaven op Dodona vandaag al beschikbaar in de programmeertalen Java en Python, maar dan als afzonderlijke en onafhankelijke oefeningen.

Een ander aspect is de beoordeling van een oefening.
Voor de Lotto-oefening is de beoordeling niet triviaal, door het gebruik van niet-deterministische functies.
Het volstaat voor dit soort oefeningen niet om de uitvoer geproduceerd door de oplossing te vergelijken met een op voorhand vastgelegde verwachte uitvoer.
De geproduceerde uitvoer zal moeten gecontroleerd worden met code, specifiek gericht op deze oefening, die de verwachte vereisten van de oplossing controleert.
Deze evaluatiecode moet momenteel voor elke programmeertaal en dus elke judge opnieuw geschreven worden.
In de context van de Lotto-oefening controleert deze code bijvoorbeeld of de gegeven getallen binnen het bereik liggen en of ze gesorteerd zijn.
Toch is deze evaluatiecode niet inherent programmeertaalafhankelijk: controleren of een lijst gesorteerd is, heeft weinig te maken met de programmeertaal van de oplossing.

\subsection{Implementeren van judges}\label{subsec:implementeren-van-judges}

Een tweede nadeel aan de huidige werking zijn de judges zelf: voor elke programmeertaal die men wil aanbieden in Dodona moet een nieuwe judge ontwikkeld worden.
Ook hier is er dubbel werk: dezelfde concepten en features, die eigenlijk programmeertaalonafhankelijk zijn, moeten in elke judge opnieuw geïmplementeerd worden.
Hierbij denken we aan bijvoorbeeld de logica om te bepalen wanneer een beoordeling juist of fout is.

\subsection{Onderzoeksvraag}\label{subsec:onderzoeksvraag}

Het eerste nadeel wordt beschouwd als het belangrijkste nadeel en de focus van deze masterproef.
Het nadeel valt te formuleren als de onderzoeksvraag waarop deze masterproef een antwoord wil bieden:

\begin{quote}
    Is het mogelijk om een judge zo te implementeren dat de opgave en beoordelingsmethoden van een oefening slechts eenmaal opgesteld dienen te worden, waarna de oefening beschikbaar is in alle programmeertalen die de judge ondersteunt?
    Hierbij is het wenselijk dat eens een oefening opgesteld is, deze niet meer gewijzigd moet worden wanneer talen toegevoegd worden aan de judge.
\end{quote}

Als bijzaak is het ook interessant om na te gaan of het antwoord op de onderzoeksvraag een voordeel kan bieden voor het implementeren van judges zelf.

De aandachtige lezer zal opmerken dat de opgave voor de Lotto-oefening programmeertaalspecifieke en taalspecifieke elementen bevat.
Zo is de beschrijving van de opgave (de voorbeelden en de gegevenstypes) in Python en zijn de namen van functies en parameters in het Nederlands.
Het ondersteunen van opgaves met programmeertaalonafhankelijke voorbeelden en vertalingen wordt voor deze masterproef expliciet als \english{out-of-scope} gezien en zal niet behandeld worden, zij het in \cref{subsec:programmeertaalonafhankelijke-opgaven}.

\section{Opbouw van de masterproef}\label{sec:opbouw}

\Cref{ch:tested} handelt over het antwoord op bovenstaande onderzoeksvraag, waar een prototype van een dergelijke judge wordt voorgesteld.

In \cref{ch:nieuwe-oefening} wordt een reeks oefeningen met verschillende functionaliteiten besproken om te illustreren hoe een testplan opgesteld moet worden en om aan te tonen welke soorten oefeningen met \tested{} gebruikt kunnen worden.
Daarna volgt een gedetailleerde beschrijving van hoe een programmeertaal geconfigureerd moet worden voor dit prototype in \cref{ch:nieuwe-taal}.
Deze twee hoofdstukken hebben ten doel een handleiding te vormen voor de lesgevers die oefeningen willen opstellen en voor de programmeurs die een nieuwe programmeertaal aan \tested{} willen toevoegen.
Daarom sluiten deze hoofdstukken qua vorm en stijl meer aan bij een handleiding, wat zich bijvoorbeeld manifesteert als een toename in het aantal codefragmenten.

Tot slot volgt met een hoofdstuk over beperkingen van het huidige prototype, en waar er verbeteringen mogelijk zijn (het "toekomstige werk").

Na de bronvermeldingen volgen nog enkele appendices waar onder andere voor twee eenvoudige oefeningen de opgave, het testplan, een oplossing en de door \tested{} gegenereerde code getoond worden.

De broncode voor \tested{} is beschikbaar op \url{https://github.com/dodona-edu/universal-judge}.