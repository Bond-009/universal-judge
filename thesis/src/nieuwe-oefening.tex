\chapter{Case-study: nieuwe oefening}\label{ch:nieuwe-oefening}

\lettrine{I}{n dit hoofdstuk} behandelen we het toevoegen van drie oefeningen in handleidingsstijl.
Elke oefening gebruikt een andere evaluatiemethode uit \cref{sec:evalueren-van-een-oplossing2}.
We gaan er van uit dat de map voor de repo voor deze oefeningencollectie al bestaat.
Deze moet voldoen aan de mappenstructuur voor oefeningen, opgelegd door Dodona\footnote{Hier beschikbaar: \url{https://dodona-edu.github.io/en/references/exercise-directory-structure/}}.

\section{ISBN}\label{sec:isbn}

Bij de eerste oefening gebruiken we enkel ingebouwde evaluators.

\subsection{Voorbereiding}\label{subsec:voorbereiding}

We beginnen met het maken van een nieuwe map \texttt{isbn} in onze map voor de oefeningencollectie.
Vervolgens maken we in deze map een configuratiebestand \texttt{config.json} voor onze oefening, met deze inhoud:

\inputminted{json}{../../exercise/isbn/config.json}

In dit bestand doen we drie belangrijke dingen:
\begin{enumerate}
    \item We geven onze oefening een naam.
    \item We duiden aan dat het om een Python-oefening gaat (zie \cref{ch:beperkingen-en-toekomstig-werk} voor waarom we dit doen).
    \item We zeggen dat ons testplan als naam \texttt{plan.json} heeft.
\end{enumerate}

\subsection{Opgave}\label{subsec:opgave}

In deze stap stellen we de opgave op.
Om dit deel kort te houden, gaan we er van uit dat de opgave al bestaat.
Maak een map \texttt{description} in onze oefeningenmap, en kopieer de opgavebestanden naar deze map.
Uiteindelijk moet deze map er zo uitzien:

\begin{minted}{text}
isbn
└── description
    ├── description.en.html
    ├── description.nl.html
    └── media
        └── ISBN.gif
\end{minted}

Om toch een idee te krijgen van waarover de oefening gaat, is hieronder een samenvatting van de opgave:

\begin{quote}
    \markdownInput{generated/isbn_description.md}
\end{quote}


\section{Testplan}\label{sec:isbn-testplan}

TODO
