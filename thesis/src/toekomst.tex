\chapter{Beperkingen en toekomstig werk}\label{ch:beperkingen-en-toekomstig-werk}

Wat kunnen we al en vooral wat niet?
Waar kan nog aan gewerkt worden?

Korte samenvatting

\section{Performance}

-> Uitleg over eerste implementatie met jupyter kernels

-> Uitleg over verschillende stadia van codegeneratie (alles apart -> zoveel mogelijk samen)

\section{Functies}

-> Dynamisch testplan
--> Dingen meerdere keren uitvoeren
--> Dingen wel of niet uitvoeren op basis van vorige uitkomst

Opm. Charlotte: Vermoedelijk zal je het bespreken in hoofdstuk 5, maar met een nauwe interpretatie van "resultaat" is dit wel mogelijk. Een idee zou bijvoorbeeld kunnen zijn om elke testcase/test/context/tab een uniek id te geven (in de JSON) en dan bij elke testcase/test/context/tab ook een lijst van dependencies te kunnen geven. Resultaat zou dan juist of fout zijn, en testen zouden enkel opgestart worden wanneer dependencies allemaal juist zijn.

Maar zeker buiten de scope van de thesis.


-> Functies/assignments
--> Functie als functie-argumenten zonder tussenstap met assignments

-> Meertaligheid

-> Beschrijving van de oefening (programmeertalen)
